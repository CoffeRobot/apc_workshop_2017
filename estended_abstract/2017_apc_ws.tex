%%%%%%%%%%%%%%%%%%%%%%%%%%%%%%%%%%%%%%%%%%%%%%%%%%%%%%%%%%%%%%%%%%%%%%%%%%%%%%%%
%2345678901234567890123456789012345678901234567890123456789012345678901234567890
%        1         2         3         4         5         6         7         8

\documentclass[letterpaper, 10pt, conference]{ieeeconf}                             
\IEEEoverridecommandlockouts         % Needed if you want to use the \thanks command
                                                     
\overrideIEEEmargins                                                            


\usepackage{times} % assumes new font selection scheme installed
\usepackage{amsmath}
\usepackage{amssymb,longtable,calc}
\usepackage{mathptmx}
\usepackage[T1]{fontenc}                                                        
\usepackage[utf8]{inputenc}                                                     
\usepackage[english]{babel}                                                     
\usepackage{epsfig}                                                             
\usepackage{subfigure}                                                          
\usepackage{textcomp} %<- allows to use \textdegree but may overwrite           
                      %other settings                                           
\usepackage[textwidth=2cm,colorinlistoftodos,disable]{todonotes} %add disable   
     
										                         
\usepackage{tikz}                                                               
\usetikzlibrary{arrows,positioning,fit,shapes,calc}
\usetikzlibrary{matrix}
\usepackage{flushend}                                                           
\usepackage{hyperref}  
\usepackage{amsmath}                                                         
% \usepackage{multirow}   

\usepackage{algorithm}    
\usepackage{algorithmic}

\usepackage{pgfplots} 
\usepackage{pgfplotstable}

\usepackage{cite}

% helper packages to work on the draft
\usepackage[tikz]{bclogo}
\usepackage{lipsum}

\usepackage{standalone}

\pgfplotsset{compat=newest}
\pgfplotsset{ 
  tick label style={font=\footnotesize}, 
  label style={font=\footnotesize}, 
  legend style={font=\footnotesize},
  title style = {font=\small}
}
\pgfplotscreateplotcyclelist{line style}{% 
  solid, mark options = {scale = .75}, every mark/.append style={fill=gray},mark=*\\% 
  densely dashed,mark options = {scale = .75},every mark/.append style={solid,fill=gray},mark=*\\% 
  densely dotted,mark options = {scale = .75},every mark/.append style={solid,fill=gray},mark=*\\% 
  dashed,mark options = {scale = .75},every mark/.append style={solid,fill=gray},mark=*\\% 
  dotted,mark options = {scale = .75},every mark/.append style={solid,fill=gray},mark=*\\% 
}
\pgfplotscreateplotcyclelist{bar style}{% 
  solid, fill=black!60!white\\%
  solid, fill=black!45!white\\%
  solid, fill=black!35!white\\%
  solid, fill=black!25!white\\%
}


\usepackage{xspace}
\makeatletter                                                                   
\DeclareTextCommandDefault{\textregisteredalt}{\footnotesize\textcircled{%      
      \check@mathfonts\fontsize\sf@size\z@\math@fontsfalse\selectfont R}}       
\DeclareRobustCommand\onedot{\futurelet\@let@token\@onedot}                     
\def\@onedot{\ifx\@let@token.\else.\null\fi\xspace}                             
\def\eg{e.g\onedot}                                                             
\def\ie{i.e\onedot}                                                             
\def\vgl{see }                                                                  
\def\Fig{Fig\onedot }                                                           
\def\Tab{Tab\onedot }                                                           
\def\Eq{Eq\onedot }
\def\Sec{Sec\onedot}                                                            
\def\etc{etc\onedot}                                                            
\def\etal{\textsl{et al}\onedot}                                                
\def\argmin{\mathop{\rm arg\,min}}                                              
\makeatother
                                                                    
\definecolor{lightGray}{rgb}{0.0,0.0,0.0}
                                                                                
\title{\LARGE \bf The Importance of Being Bryce}                                                                               
                                                                                
                                                                                
\author{Hang Kayu ~~ Francisco Vina ~~ Michele Colendachise ~~ Karl Pauwels ~~ Alessandro Pieropan ~~ Danica Kragic%
\thanks{}%
\thanks{}
\thanks{}}

\begin{document}                                                                
                                                                                
\maketitle                                                                      
\thispagestyle{empty}                                                           
\pagestyle{empty}



%%%%%%%%%%%%%%%%%%%%%%%%%%%%%% ABSTRACT %%%%%%%%%%%%%%%%%%%%%%%%%%%%%%%%%%%%%
\begin{abstract}

In this paper we will present our framework used in the Amazon Picking Challenge in 2015 and some lessons-learned that may prove useful to researchers and future teams participating in the competition. The competition proved to be a very useful occasion to integrate the work of various researchers at the Robotics, Perception and Learning laboratory of KTH, measure the performance of our robotics system and define the future direction of our research.

\end{abstract}

%%%%%%%%%%%%%%%%%%%%%%%%%%%%%% INTRODUCTION %%%%%%%%%%%%%%%%%%%%%%%%%%%%%%%%%%%
% \begin{}
\section{INTRODUCTION}
\label{sec:introduction}

There are three main criteria engineers evaluates when determining the need of robots in certain applications: dirty, dull and dangerous. Those are known as the 3D of Robotics. The application proposed by the Amazon Picking Challenge meets certainly the second criteria as picking and placing objects in boxes could be a very repetitive and boring job. However, despite the defined and controlled environment the application of robots is still very challenging due to the nature of the objects to handle.
In this work we present the framework we develop at the Robotic, Perception and Learning lab (RPL) in 2015 with the purpose of sharing the lessons learned with the community.
First we will describe the platform used in the competition in Sec.\ref{sec:platform} to motivate the strategy we adopted in Sec.\ref{sec:strategy}. Then we will describe the core of our system that controls the whole pipeline of actions using behavioural trees in Sec.\ref{sec:trees}. Then we will describe our perception module starting with the localization of the shelf in Sec.\ref{sec:shelf} and detections of the objects \ref{sec:vision}. Finally we will describe our grasping strategy in Sec.~\ref{sec:grasping} and draw some conclusion about the limitation of our system in Sec.\ref{sec:conclusion}.

%% %%%%%%%%%%%%%%%%%%%%%%%%%%% PR2 %%%%%%%%%%%%%%%%%%%%%%%%%%%%%%%% 
\section{Platform}
\label{sec:platform}

We used the PR2 research robot from Willow Garage shown in Fig.
\ref{fig:PR2} for the 2015 APC competition. The robot consists of two
7-DOF manipulators attached to a height-adjustable torso and equipped with parallel-jaw
grippers. A pan-tilt controllable robot head is located on the upper part of the robot
and equipped with a set of RGB cameras and a Kinect camera which we used in
our object segmentation system. The robot is also
equipped with a tilt-controllable laser scanner located right beneath the head which we
also used for object segmentation purposes.
A set of 4 wheels at the base provides the robot with omnidirectional mobility, which we
exploited in order to control the robot back and forth from the shelf to pick up objects
and release them in the target box.


\begin{figure}[h]
\centering
\includegraphics[trim=15.0cm 4.0cm 6.0cm 0cm,clip=true,scale=0.07]{figures/pr2.jpg}
\caption{Team CVAP's PR2 robot platform used for the Amazon Picking Challenge 2015 in Seattle, USA.}\label{fig:baxter}
\end{figure}

We provided some minor hardware modifications to the PR2 robot in order to
address some of the challenges  of the picking task,
namely custom-made extension fingers for the parallel gripper in order
to be able to reach further inside of the bins of the shelf and a high resolution
webcamera which we attached on the PR2's head in order to get
a richer set of image features for our Simtrack vision system to detect the target 
objets in the shelf.

The robot ran on an Ubuntu 12.04 computer with a real-time linux kernel that provides
1 KHz manipulator control. All the high level task execution, perception, grasping and manipulation software components were developed under the Robot Operating System (ROS).


%% %%%%%%%%%%%%%%%%%%%%%%%%%%% PR2 %%%%%%%%%%%%%%%%%%%%%%%%%%%%%%%% 
\section{Strategy}
\label{sec:strategy}



%% %%%%%%%%%%%%%%%%%%%%%%%%%%% BTS %%%%%%%%%%%%%%%%%%%%%%%%%%%%%%%%
\section{Behavioral Trees}
\label{sec:trees}



%% %%%%%%%%%%%%%%%%%%%%%%%%%%% SHELF %%%%%%%%%%%%%%%%%%%%%%%%%%%%%%%% 
\section{Shelf Localization and Base Movement}
\label{sec:shelf}

As described in Sec.~\ref{sec:platform}, we used a PR2 as our robot platform. Since the arms' reachability of PR2 is relatively small in comparison to the shelf size, it is not feasible to define a fixed location for the robot to achieve the required task. As such, we have to exploit the mobility to enlarge the working space, so that the robot would be able to reach and grasp from most of the shelf bins, as well as loading the grasped objects into the order bin.

For this, shelf localization, which serves as the only landmark in the workcell, is essential for our system to guide the robot navigating between different grasping positions. Since the robot movement accumulates localization errors, it is necessary to localize the shelf in real time to close the control loop for base movement. 

\begin{figure}[htb]
\centering
	\includegraphics[height=0.39\columnwidth]{figures/localization.png}
	\includegraphics[height=0.39\columnwidth]{figures/binFrames.png}
        \caption{\emph{Left}: An example of shelf localization shown in rviz. The x and y coordinates of shelf\_frame is defined as the center of two front legs, while the height of it is the same as the base\_laser\_link. \emph{Right}: The bin frames are located at the right bottom corner of each bin.}
        \label{fig:localization}
\end{figure}

As shown in Fig.~\ref{fig:localization}, we use the base laser scanner to localize the two front legs of the shelf. Observing that the shelf is located in front of the robot, where no any other objects are close by. Therefore, it is reasonable to find the closest point cluster and consider it as one of the legs, while the remaining cluster is considered as another leg. However, this is not a reliable procedure as there could be noise or other unexpected objects, e.g., human legs. As such, our shelf localization consists of two procedures as follows:

\begin{itemize}
 \item \textbf{Detection:} Once the front legs are detected, before the system starts to autonomously work on assigned tasks, a human supervisor needs to confirm to the robot that the detection is correct. In case when the detection is incorrect, we need to clear the occlusions in front of the shelf until a confirmation is made.
 \item \textbf{Tracking:} While the robot is moving, given that we know the motion model of the robot, we update the shelf localization using a Kalman filter.
\end{itemize}

Having localized the shelf, we further estimate the shelf bin frames based on the known mesh model, see Fig.~\ref{fig:localization}. As will be described in Sec.~\ref{fig:localization}, knowing the bin frames will facilitate the grasp planning in our system.

%% %%%%%%%%%%%%%%%%%%%%%%%%%%% VISION %%%%%%%%%%%%%%%%%%%%%%%%%%%%%%%%
\section{Vision}
\label{sec:vision}



%% %%%%%%%%%%%%%%%%%%%%%%%%%%% GRASPING %%%%%%%%%%%%%%%%%%%%%%%%%%%%%%%%
\section{Grasping}
\label{sec:grasping}



\section{Conclusion}
\label{sec:conclusion}

We presented the framework used in the competition in 2015 and all the challenges we had to face. With hindsight we would have brought our own robot to the competition since many of the issue we had to address were related to the robot provided at the venue, leaving us no time to calibrate and tune our framework. Moreover the team lacked in mechanical expertise restricting our choice of the platform to the PR2, the most suited robot we had in our laboratory at the time. I would have been nice to design a specialized robot for the competition as other teams did. 

\end{document}
